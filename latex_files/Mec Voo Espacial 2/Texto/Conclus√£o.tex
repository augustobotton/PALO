A partir da experiência educativa e prática proporcionada por este trabalho, aprofundamos nossos conhecimentos em mecânica orbital, engenharia de sistemas espaciais e programação em Python. Embora tenhamos enfrentado alguns desafios significativos ao longo do projeto, conseguimos superá-los e obter resultados significativos. A migração do Matlab para o Python, apesar de ter apresentado desafios, proporcionou a oportunidade de desenvolver habilidades valiosas em programação e pensamento crítico.

Por meio das adaptações, conseguimos verificar a importância do pensamento computacional, que não se resume apenas à programação, mas também envolve a resolução de problemas, o design de sistemas e a compreensão do comportamento humano. Reconhecemos a necessidade de eficiência e clareza na codificação, bem como a importância de comentários claros e concisos para permitir uma fácil compreensão e futuras revisões.

Além disso, este trabalho nos permitiu apreciar a complexidade do processo de projeto de um veículo lançador e a quantidade significativa de variáveis e parâmetros que devem ser considerados. A otimização manual revelou-se uma tarefa complexa e demorada, reforçando a necessidade de ferramentas automatizadas de otimização.

Para futuras investigações, um foco maior na compreensão e otimização da codificação em Python seria extremamente benéfico, bem como a implementação de algoritmos avançados de otimização. Deste modo, será possível refinar ainda mais o projeto do veículo lançador, melhorando a eficiência e a eficácia das missões espaciais.

Em suma, apesar das limitações e desafios, esta experiência proporcionou um aprendizado significativo e abriu caminho para futuras explorações e melhorias nesta área fascinante da engenharia espacial. A experiência adquirida servirá de base para futuros estudos e aplicações, consolidando nossa compreensão sobre mecânica de voo orbital e nos deixando mais preparados para enfrentar os desafios que o futuro da exploração espacial nos reserva.