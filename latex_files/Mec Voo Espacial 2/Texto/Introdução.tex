\chapter{Introdução}

\par Este trabalho tem como foco a inserção em órbita geossíncrona de um CubeSat 8U. A primeira etapa desse trabalho é estudar e adaptar um foguete existente para atender aos requisitos de variação de velocidade (∆v), conforme determinado pela equação do foguete. Esta adaptação inclui a verificação da distribuição ideal de massa entre os estágios do foguete, propondo alterações se necessário para otimizar o impulso e, assim, alcançar o ∆v necessário.

O foguete será analisado em termos de voo ascendente, abordando aspectos como inserção em órbita, avaliação da factibilidade em termos de impulso de velocidade, ajuste de distribuição de massa para capacitação de inserção orbital, cálculo de parâmetros inerciais, e avaliação da factibilidade em termos de energia específica, órbita de referência e altitude.

Ademais, será abordado o voo ascendente de foguete em termos de voo balístico e trajetória de rotação gravitacional, bem como o modelo de trajetória plana de voo ascendente. Ajustes de parâmetros para obtenção da órbita desejada, considerando excentricidade e inclinação, serão discutidos.

Além disso, é apresentado o modelo do veículo de inserção orbital e o conceito de rastro planetário, também conhecido como "ground track". Ao final deste trabalho, o objetivo é ter um modelo de foguete capaz de simular a inserção orbital do CubeSat 8U em órbita geossíncrona, desde o lançamento até a aquisição da órbita alvo.