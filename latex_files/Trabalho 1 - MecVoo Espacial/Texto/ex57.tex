\subsection{Exemplo 5.7}

Um veículo espacial em uma órbita terrestre elíptica, com a = 6.900 km, e = 0, 6, $\Omega$ = 120\degree, $\omega$ = 25\degree \ e $i$ = 10\degree. Quando o veículo está no apogeu, um impulso de velocidade é aplicado com um ângulo $\beta$ = 100\degree , relativo ao vetor velocidade, medido no sentido anti-horário em um plano normal à órbita inicial. A magnitude deste impulso é tal que não há alteração da magnitude da velocidade orbital. Determine a nova órbita do veículo espacial.

\par O formato das órbitas será o mesmo, dado que o impulsivo foi aplicado em um plano normal a órbita inicial, não houve alteração da magnitude da velocidade e a distância radial no ponto de manobra não foi alterada, dado as características da manobra. 

\begin{table}[h]
\centering
\caption{Elementos Orbitais da Orbita Resultante - Exemplo 5.7}
\label{tab ex57}
\begin{tabular}{|c|c|c|}
\hline
Elementos Orbitais Finais & Valor               & Unidade \\ \hline
Semi-Eixo Maior                         & 6899999.999         & m       \\ \hline
Excentricidade                        & 0.6000000000000002  & -       \\ \hline
$\Omega$                  & -14.509563087754747 & grau    \\ \hline
$\omega$                     & 158.77278698980075  & grau    \\ \hline
inclinação                       & 11.694220111804377  & grau    \\ \hline
$\alpha$                     & 20.000              & grau    \\ \hline
\end{tabular}%

\end{table}

\par A diferença entre a inclinação inicial e final é de $\delta_i=1.694$. Já o ângulo $\alpha=20\degree$. Como a medida do ângulo $\alpha$ é feita da linha dos nodos na intersecção dos planos orbitais e a medida da inclinação é feita a partir da linha dos nodos da órbita com o plano equatorial, tem-se essa diferença dos ângulos. Com isso conclui-se que uma aplicação de impulso de velocidade não implica em uma alteração da inclinação da nova órbita.