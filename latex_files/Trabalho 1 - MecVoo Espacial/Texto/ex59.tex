\subsection{Exemplo 5.9}

Enunciado: Calcule os impulsos de velocidade e o tempo requerido para uma transferência de Hohmann a partir de uma órbita circular terrestre de altitude $250 km$ (órbita de estacionamento - parking orbit), para uma órbita geosíncrona.

\begin{table}[h]
\centering
\caption{}
\label{tab: ex5.9}
\begin{tabular}{|c|c|c|}
\hline
                       & Valor      & Unidade \\ \hline
$\delta_{v1}$                 & $2440.0824$  & $m/s $    \\ \hline
$\delta_{v2}$                 & $1472.0333$  & $m/s$     \\ \hline
Tempo da Transferência & $18961.0618$ & $s$       \\ \hline
\end{tabular}
\end{table}


A transferência de Hohmann é uma manobra orbital amplamente utilizada na engenharia espacial para mover uma espaçonave entre duas órbitas circulares ao redor de um corpo celeste, como a Terra. A manobra é feita com dois impulsos conforme listado abaixo:

\begin{enumerate}

    \item Inserção em órbita de transferência: A espaçonave é colocada em uma órbita elíptica ao redor do corpo celeste de partida. Essa órbita tem um periastro (ponto mais próximo do corpo celeste) na órbita original e um apoastro (ponto mais afastado do corpo celeste) na órbita de destino desejada.

    \item Inserção na órbita de destino: Quando a espaçonave atinge o apoastro da órbita de transferência, uma segunda queima de propulsão é realizada para alterar sua velocidade e direção, de forma a circularizar a órbita na qual deseja-se chegar.

\end{enumerate}

O valor do incremento da velocidade em cada impulso e o tempo para que ocorra a transferência estão dados na tabela \ref{tab: ex5.9}.