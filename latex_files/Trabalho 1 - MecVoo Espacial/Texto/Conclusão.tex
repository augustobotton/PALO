\chapter{Conclusão}

\par Este trabalho proporcionou uma exploração abrangente e prática da mecânica de voo orbital, destacando a importância da modelagem e do cálculo preciso das trajetórias orbitais. Através da utilização de códigos em Python, foi possível demonstrar e validar os conceitos teóricos em um contexto aplicado, mostrando a eficácia e a precisão das soluções proposta
\par Os resultados obtidos ao longo deste trabalho contribuem significativamente para o entendimento da engenharia aeroespacial, fornecendo uma compreensão aprofundada da mecânica de voo orbital e evidenciando o potencial da programação computacional como uma ferramenta eficaz para resolver problemas complexos na área. A capacidade de modelar e calcular trajetórias orbitais de forma precisa e confiável é fundamental para o desenvolvimento de tecnologias espaciais avançadas, como satélites e veículos espaciais tripulados.
\par Algumas discrepâncias de resultados foram encontradas, o que torna necessário um aprofundamento dos métodos de solução e um cuidado durante a aplicação desses resultados devido aos erros inerentes ao cálculo numérico. Contudo o trabalho cumpriu sua função no entendimento da mecânica de voo espacial para os tópicos estudados até o momento.