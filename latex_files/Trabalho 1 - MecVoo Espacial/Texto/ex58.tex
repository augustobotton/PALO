\subsection{Exemplo 5.8}

Foi solicitado o menor impulso total requerido, em uma manobra de transferência orbital bi impulsiva, de uma órbita circular terrestre de 500 km de altitude, para a órbita elíptica vista no exemplo 5.7 que intercepta a circular.
\par Utilizando o script feito em aula, obtém-se os seguintes valores mínimos:

\begin{table}[h]
\centering
\caption{Resultados Exemplo 5.8}
\label{ex58}

\begin{tabular}{|c|c|c|}
\hline
                              & Valor      & Unidade \\ \hline
$v_{at}$                         & 5264.8753  & m/s     \\ \hline
$v_pt$                         & 8450.5766  & m/s     \\ \hline
$v_i$                          & 7612.6039  & m/s     \\ \hline
$v_f$                          & 3800.2669  & m/s     \\ \hline
$\delta_v1 $    & 837.9726   & m/s     \\ \hline
$\delta_v2 $    & -1464.6083 & m/s     \\ \hline
$\delta_{vt}$ & 2302.5810  & m/s     \\ \hline
\end{tabular}

\end{table}

\par Como pode ser visto na tabela acima o menor impulso total requerido é de 2303.581 m/s. 