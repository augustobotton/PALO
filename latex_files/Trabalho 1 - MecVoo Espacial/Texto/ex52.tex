\par Nessa seção será resolvido um exemplo onde serão obtidos os parâmetros de uma órbita em um problema de dois corpos a partir de uma única observação de posição e velocidade. Os elementos orbitais consistem em 6 parâmetros, que descrevem a progressão e a orientação da órbita. Um conjunto comum de elementos orbitais inclui a excentricidade, semi eixo maior, tempo de periastro, longitude celeste do nodo ascendente, inclinação e argumento de periastro. 
\par Do enunciado tem-se que uma espaçonave observada no referencial centrado na Terra possui a seguinte posição e velocidade celeste: \textbf{r}=-500\textbf{I}+12500\textbf{K} [km] e \textbf{v}=5\textbf{I}-8\textbf{J}[km/s]. Foi solicitado que fosse determinado os parâmetros orbitais. Utilizando o programa desenvolvido em aula, foram obtidos os valores mostrados na tabela \ref{ex52}

\begin{table}[H]
\centering
\caption{Resultados Exemplo 5.2 -Determinação de Órbita}
\begin{tabular}{|c|c|c|}
\hline
Elementos Orbitais & Valor               & Unidade \\ \hline
Semi-Eixo Maior          & -13382.403826218939 & Km      \\ \hline
Excentricidade          & 1.9765961447821856  & - \\ \hline
$\tau$        & 416.7937786907604   & s       \\ \hline
$\Omega$      & 122.0053832080835   & graus   \\ \hline
i          & 71.26309861909091   & graus   \\ \hline
$\omega$      & 95.71519588364482   & graus   \\ \hline
\end{tabular}
\label{ex52}
\end{table}

\par Analisando os elementos pode-se afirmar que a espaçonave se aproxima da Terra, uma vez que $\tau$ é positivo. Isso também pode ser visto devido a anomalia verdadeira no quarto quadrante.

\nocite{book:226549}