\chapter{Introdução}

\par A mecânica de voo orbital é um campo crucial dentro da engenharia aeroespacial, que tem visto um crescente interesse em função do aumento da exploração espacial e do desenvolvimento de tecnologias de satélites. Embora a teoria subjacente seja bem estabelecida, a natureza complexa do ambiente espacial e as numerosas variáveis envolvidas tornam a modelagem e o cálculo das trajetórias orbitais um desafio contínuo. Este trabalho busca desenvolver uma compreensão mais aprofundada desta área, focando na resolução de exercícios práticos com a ajuda de códigos em Python, a fim de demonstrar e validar os conceitos teóricos em um contexto aplicado.

\par O trabalho se divide em vários tópicos principais, cada um deles tratando um aspecto específico da mecânica de voo orbital. A abordagem adotada envolve a identificação dos principais desafios em cada área, a formulação de soluções através da codificação em Python e a subsequente validação dessas soluções. Essa abordagem pragmática e orientada para soluções permite uma compreensão mais profunda e aplicável dos conceitos, ao mesmo tempo em que demonstra o valor da programação como ferramenta na resolução de problemas complexos em engenharia aeroespacial.

\par Os resultados obtidos ao longo deste trabalho confirmam o sucesso da abordagem computacional. As soluções propostas para os exercícios demonstram não apenas a viabilidade da utilização de Python para resolver problemas de mecânica de voo orbital, mas também a precisão e a eficácia dessas soluções. Consequentemente, este trabalho serve como uma valiosa contribuição para a engenharia aeroespacial, oferecendo uma compreensão aprofundada da mecânica de voo orbital e destacando o potencial do Python como uma ferramenta eficaz para a solução de problemas complexos na wárea.